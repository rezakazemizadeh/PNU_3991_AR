\documentclass{book} 
\usepackage[top=3cm,right=3.5cm,bottom=3cm,left=3cm]{geometry}
\renewcommand{\baselinestretch}{1.7}
\parindent=5pt
\begin{document}
page$151$\\\\\par
a survey participant request unless the subject line, the content of the message, and any incentives both "hooks" and induces subjects to participate. Moreover, a recent survey by Gilbert (2001) shows that as many as 85 percent of users, at least occasionally, delete messages without reading them. A small number of users are also setting email filters to eliminate postings from all but well-known senders. These options arc making it more difficult for the e-researcher to communicate with the targeted population.\\
\textbf{Authenticity}\par  Issues of authenticity plague all survey designs and may be exacerbated online. Unsophisricated designs make it difficult, Or impossible, to determine if the participant replying to the survey is the one who was Sent the survey. Further, it may be difficult to determine if subjects have replied multiple times to the e-survey.\\
\textbf{Security and Confidentiality}\par  Issues of anonymity are also exacerbated online (at least in the minds of some potential respondents). The e-researcher is forced to create a trusting environment by addressing directly issues of confidentiality, secure storage of results, and ethical research behavior. These issues are discussed further in the ethics chapter (Chapter 5).\\
\textbf{Respondent Anger}\par  Even the most well-crafted and inviting e-survey may be perceived by some respondents as aggravating spam (unsolicited and unwanted email). The response of a recipient of a mail delivered paper survey is usually to throw it away or return it unanswered. There is a small possibility that a disgruntled recipient of an e-survey may reply with a virus or Trojan horse email bomb, or inappropriately forward, alter, or in other ways misuse your e-survey. Until authentication and digital signatures become more wide-spread, there is probably little that e-researchers can do to eliminate this problem, other than to take standard procedures for protecting and checking their email and Web sites for viruses or other malicious attacks.\\
\textbf{Procrastination}\par  The advantage of time shifting Can also encourage procrastination. Some users have noted the ease with which email can be glanced at and left unattended to at the bottom Ofa growing list of emails. Provision Of an attention-grabbing subject line is critical to reduce this disadvantage.\\
\textbf{CRITICAL ISSUES IN e-SURVEY DESIGN AND ADMINISTRATION}\par  In this section we look at several key design issues that every researcher must address when using e-surveys. The task of the e-researcher is to minimize each of these errors to the greatest degree possible within the constraints of the available time and budget.\newpage
page$152$\\\\\par
\textbf{Reducing e-Survey Error}\par  Every survey is subject to error. Even if one were able to survey each member of the target population, there may still be some error due to respondent misinterpretation of the questions or misrepresentation of themselves. However, since surveying all members is rarely possible, additional errors due to sample selection may also occur, Despite great care in selecting n sample, there will always be random variations in any population that may, quite by chance, bias even the most meticulously designed and administered survey. However, it is the responsibility of researchers to eliminate AS much of the systemic error in their design and Administration of the e-survey as possible. Next we describe the major sources of error, common to all forms of survey, with brief notations on the particular manifestation of the error in Net-based forms of Survey research. Major sources Of error reduce the value, veracity, and impact of any survey including those conducted online.\\
\textbf{Frame or Coverage Error}\par  Coverage error occurs when only a particular subset of the target population is included in the survey. The sampling frame is the list or source of names from which the sample is drawn. If this list does not contain all of the members of the population, and especially if some groups or individuals are systematically eliminated from the frame, then frame error will result in survey result errors. This is an obvious danger for e-researchers, in that the entire general population does not currently have access to the Net. Thus for the foreseeable future there will always be elements of the whole population that are eliminated from a Net-based survey due to coverage error. However, there are a growing number of target populations to whom 100 percent or close to 100 percent of the members are online. This group would include employees of many companies and members Of certain professions or social organizations. Coverage error has led some researchers to conclude that e-surveys are not useful (Dillman, 2000) for general population studies at this time. Although we agree that one cannot make inferences about the whole population based on the subset who use the Net, we contend that there is still a great deal of valuable information that can be obtained from sampling from the growing number of people who access the Net on A regular basis.\\
\textbf{Measurement error}\par Measurement error occurs when there is a variation between the information the researcher is looking for and that obtained from the research process. Measurement error can begin in the design process if the researcher is not clear what type of information is being sought, It is most commonly found in measurement bias within the survey itself, in the form of confirsing, uninterpretable, or biased questions producing results that are inaccurate, uninterpretable, or both.Measurement error may also occur during completion of the survey if respondents make data entry errors when completing the survey. Finally, measurement error may result from error in data analysis. Careful wording of instructions and provision of examples are useful •anays to reduce measurement error. \newpage
page$153$\\\\\par
\textbf{Nonresponse Error}\par  Nonresponse error Occurs when those who did not respond to the survey are in some ways different from those who did respond and that difference is relevant to the research study. An obvious example would be an e-survey to determine workload levels Of school principals. The principals with the heaviest workloads may be the ones least likely to take the time to complete an e-survey and thus their critical information will be lost. resulting in considerable nonresponse error.\\
\textbf{Response Bias}\par  Response bias occurs when survey respondents deliberately or inadvertently falsify or misrepresent their answers. Respondents may falsify answers to give socially acceptable answers, to avoid potential embarrassment, or to conceal personal or confidential information. Misrepresentations occur when respondents provide incorrect responses to questions to which there is a correct answer.\\
\textbf{ACHIEVING A HIGH RESPONSE RATE}  Although there is no absolute minimum for an acceptable response rate, the higher the response rate, the more accurately the survey sample results will reflect the opinions of the target population. Researchers use theories to help explain and predict a variety of communication, interaction, and other human behaviors. For example, in the field of social sciences, social exchange theory has been usefully adapted to provide guidelines for the construction and administration of surveys (Dillman, 2000). Underlying this theory is the premise that human behavior occurs and is channeled by the rewards that result from these behaviors. If the behavior is to continue, the rewards to the individual must exceed the costs of engaging in the behavior. Further, since the rewards may be long term or delayed in arriving, the participant must have trust (in the researcher) that the benefits will outweigh the costs. In the following section, we describe the general means by which these three important variables—rewards, risks, and trust—can be used by the researcher to increase the response rates of e-surveys.\\
\textbf{ Rewards}\par There are a variety of techniques by which the e-researcher can enhance the respondents' perception Of reward for participating in the e-survey. Most obviously, the e-researcher may wish to build in tangible incentives such as gift certificates, promises of cash, discounts. or prizes. Reward is also engendered by the respondents' perception that the survey is useful and worthwhile and that their participation in the survey is important. Efforts should also be made to validate the position Of respondents by acknowledging their inclusion in the important group selected for this study. Engaging participants immediately in the text of a cover letter and in the first few questions is vitally important to this perception of reward. Engagement is
\end{document}