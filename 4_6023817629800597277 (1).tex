\documentclass [7pt]{beamer}
\usepackage{xcolor}	
\usepackage{tikz}
\usetheme{Frankfurt}
\useoutertheme{infolines}
\usepackage{ragged2e}
\begin{document}
\small
\section*{kholase safahat 151..153 }
\subsection*{reza kazemi zadeh }	
\begin{frame}
\justifying	
\textbf{Authenticity} 
Or impossible, to determine if the participant replying to the survey is the one who was Sent the survey. Further, it may be difficult to determine if subjects have replied multiple times to the e-survey.

 
\textbf{Security and Confidentiality} 
secure storage of results, and ethical research behavior. 


\textbf{Respondent Anger} 
e-researchers can do to 
eliminate this problem, other than to take standard procedures for protecting and checking their 
email and Web sites for viruses or other malicious attacks.

 
\textbf{Procrastination} 
The task of the e-researcher is to minimize each of these errors to the greatest degree possible within the constraints of the available time and budget.
\end{frame}


\begin{frame}
\justifying
\textbf{Reducing e-Survey Error} 
Every survey is subject to error. 
Major sources Of error reduce the value, veracity, and impact of any survey including those conducted 
online.

 
\textbf{Frame or Coverage Error} 
Coverage error occurs when only a particular subset of the target population is included in the survey. Coverage error has led some researchers to conclude 


\textbf{Measurement error} 
Measurement error occurs when there is a variation between the information the researcher is looking for and that obtained from the research process. Finally, measurement error may result from error in data analysis. Careful wording of instructions and provision of examples are useful arrays to reduce measurement error. 
\end{frame}	

\begin{frame}
\justifying 
\textbf{Nonresponse Error} 
Nonresponse error Occurs when those who did not respond to the survey are in some ways different from those who did respond and that difference is relevant to the research study.

 
\textbf{Response Bias} 
Response bias occurs when survey respondents deliberately or inadvertently falsify or misrepre-sent their answers. Respondents may falsify answers to give socially acceptable answers, to avoid

 
\textbf{Rewards} There are a variety of techniques by which the e-researcher can enhance the respondents' perception Of reward for participating in the e-survey. Most obviously, the e-researcher may wish to build in tangible incentives such as gift certificates, promises of cash, discounts. or prizes. 

\end{frame}
\end{document}
